\section{Experimental Result:}
The developed image to text conversion system using artificial intelligence was implemented and evaluated on various datasets. The experimental results demonstrate the effectiveness and efficiency of the system in recognizing and extracting text from images, and highlight the potential benefits of combining deep learning techniques for improved performance. The results provide insights into the strengths and weaknesses of the system, and demonstrate its potential for practical applications such as document scanning, optical character recognition, and image captioning.
\subsection{Introduction:}
In this section, the experimental results of the developed image to text conversion system using artificial intelligence are presented and analyzed. The system was implemented and evaluated on various datasets to assess its effectiveness and efficiency in recognizing and extracting text from images. The experiments were designed to evaluate the performance of the system on different types of images and text, and to compare its performance with existing state-of-the-art systems.The experimental results provide insights into the strengths and weaknesses of the developed system and highlight the potential benefits of combining deep learning techniques such as CNNs, RNNs, and attention mechanisms for improved performance. The results also demonstrate the potential of the system for practical applications, such as document scanning, optical character recognition, and image captioning.Overall, the experimental results provide a comprehensive evaluation of the developed system and its performance on various datasets, providing a foundation for future research and development in the field of image to text conversion using artificial intelligence. The next section will provide a detailed analysis of the experimental results and their implications.  
\subsection{Analysis:}
The experimental results of the developed image to text conversion system using artificial intelligence demonstrate its effectiveness and efficiency in recognizing and extracting text from images. The system achieved high accuracy rates on various datasets, outperforming existing state-of-the-art systems. The experiments conducted on the system demonstrated the benefits of combining deep learning techniques such as CNNs, RNNs, and attention mechanisms for improved performance. The results showed that the use of attention mechanisms improved the system's performance in recognizing and extracting text from complex images with multiple text regions. The experimental results also highlighted the limitations of the system, particularly in recognizing text from low-quality images with noise and distortion. The results suggest that further research is needed to develop more robust and resilient systems that can handle such challenges. The experimental results demonstrate the potential of the developed system for practical applications such as document scanning, optical character recognition, and image captioning. The findings provide valuable insights into the strengths and weaknesses of the system, and provide a foundation for future research and development in the field of image to text conversion using artificial intelligence

\subsection{Application}
Image-to-text conversion technology has various applications, including image indexing, content-based image retrieval, and image search engines. Our model can be used to automatically generate text captions for images, which can be useful for image indexing and content-based image retrieval. It can also be used to build an image search engine that allows users to search for images using text queries.
\subsection{Conclusion}

In this project, we built a deep learning model for image-to-text conversion and evaluated its performance using the BLEU score. Our model achieved a high score and was able to accurately generate text captions for various images. This technology has various applications, including image indexing, content-based image retrieval, and image search engines.