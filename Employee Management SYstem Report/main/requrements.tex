\title{requirements}
\begin{center}
    \textbf{Chapter 04}\\
  \section{  \large \textbf{Implementation and Testing }}
\end{center}
\vspace{2.5mm}
\subsection{Introduction}
The introduction of the Employee Management System (EMS) project report outlines the
essential requirements for developing a comprehensive software solution to streamline
human resources processes. The EMS aims to address the organization's needs by
centralizing employee information, optimizing attendance tracking, leave management,
payroll processing, and performance evaluations. This section highlights the significance of
the EMS in improving HR operations, enhancing workforce productivity, and facilitating
data-driven decision-making. By providing a user-friendly interface for employees and
robust reporting tools for HR managers, the EMS ensures seamless access to critical
information and valuable insights for effective employee management. The project report
delves into these requirements to present a clear roadmap for successful EMS development
and implementation, aligning with the organization's objectives and fostering overall
efficiency and growth.
\subsection{Programming Language}
The programming language chosen for the Employee Management System (EMS) project
plays a pivotal role in its successful development and implementation. In this section of the
project report, we introduce the programming language that best aligns with the project's
objectives and requirements. After careful consideration, [Programming Language] has
been selected as the primary language for building the EMS due to its versatility,
robustness, and extensive community support. Its object-oriented nature enables modular
code development, fostering scalability and maintainability. Additionally, [Programming
Language] offers a wide range of libraries and frameworks that expedite the development
process, reducing time-to-market. The choice of [Programming Language] aims to ensure a
seamless and efficient EMS with a focus on clean code, security, and optimal performance.
The following sections of the report will delve into the technical aspects of utilizing
[Programming Language] and showcase its advantages in creating a powerful and user-
friendly Employee Management System.
\begin{itemize}
    \item MySQL
    \item Java
\end{itemize}
\subsection{Software Requirements}
\begin{itemize}
    \item Apache NetBeans
\item XAMPP Server
\end{itemize}
\subsection{Hardware Requirements}
\begin{itemize}
    \item OS: Windows 10.
    \item Ram: 4 GB.
    \item Hard Disk: 100 GB.
    \item Monitor, Keyboard, Mouse.
\end{itemize}
