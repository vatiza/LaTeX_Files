\begin{center}
\chapter{Chapter 05}
\end{center}
\vspace{11mm}
\section{Conclusion}
\subsection{Conclusion}
The Blood Bank Android Application Project Report's conclusion emphasizes the project's major
discoveries, successes, and ramifications. It highlights the relevance of the created application 
and its potential influence on the healthcare industry while summarizing the complete process
from beginning to end. The conclusion also discusses the difficulties encountered throughout the
project and how they were resolved, highlighting the team's commitment and problem-solving
skills. The project's accomplishments are noted in the conclusion, including the efficient donor-recipient matching system, user-friendly interface, and successful installation of the Blood Bank
Android Application. The application's potential to increase blood supply accessibility, shorten
reaction times in emergency situations, and help save lives is also highlighted in the conclusion.
The conclusion may also cover any ideas for application upgrades or growth, assuring the
application's viability and ongoing efficacy. The project team may thank the sponsors,
contributors, and stakeholders who helped out and worked together on it.

\subsection{Limitation and Future Works}
\subsubsection{Limitation}
During its development and implementation, the Blood Bank Android
Application Project Report ran into some restrictions. The effectiveness
of the donor-recipient matching system may be impacted by the
availability of limited data, particularly regarding registered blood
donors and recipients. Additionally, the real-time notification feature
may not work properly in places with weak network coverage due to
device compatibility issues and connectivity concerns. It is crucial to
guarantee user privacy and resolve privacy issues related to sensitive
medical data. Continuous efforts are needed to promote user adoption
and increase awareness of the advantages of the application. The range
of functionality and interaction with current healthcare systems may
have been constrained by resource limitations. It could not have been
possible to do thorough testing across a range of situations and devices,
which could have resulted in problems going undetected. The success of
the application may also be impacted by geographic restrictions on the
accessibility of blood banks and medical facilities. Recognizing these
shortcomings offers suggestions for upcoming enhancements and
adjustments that will result in a more reliable and effective Blood Bank
Android Application.
\subsubsection{Future Works}
Future work in the framework of the Blood Bank Android Application
Project Report includes a number of prospective improvements and
extensions to increase the functionality and effect of the application. The
improvement of the donor-recipient matching system using cutting-edge
algorithms and machine learning methods is one of the crucial areas of
future study. This may result in a more accurate and effective matching
of compatible donors and recipients, maximizing the use of the blood
supply. Another critical area for future development is integrating the
application with current healthcare systems, electronic health records,
and hospital databases. Such integration can improve overall
coordination in the blood donation process, expedite information flow,
and enable smooth communication between blood banks and medical
facilities. Additionally, offering reminders for blood donations and
support in multiple languages can improve accessibility, user
engagement, and donor participation. Including social elements and user
feedback systems can help contributors feel more connected to one
another and collect insightful data for ongoing improvement.
Additionally, future work can concentrate on developing offline
capabilities to guarantee that crucial functions are still accessible even in
locations with poor internet connectivity. Considering the app's growth
to other platforms, including iOS, can increase its impact and reach. The project report establishes the groundwork for ongoing growth and
improvement, ensuring that the application continues to be successful in tackling significant difficulties in the blood donation process and
favorably impacting healthcare services.

\subsection{Reference}
1. Kavitha, B.; Srimathi, C. Benchmarking on offline Handwritten Tamil Character Recognition using convolutional
neural networks. J. King Saud Univ. Comput. Inf. Sci. 2019. [CrossRef]
\newline 
\newline
2. Dewan, S.; Chakravarthy, S. A system for offline character recognition using auto-encoder networks. In Proceedings
of the International Conference on Neural Information Processing, Doha, Qatar, 12–15 November 2012.
\newline 
\newline
3. Ahmed, S.; Naz, S.; Swati, S.; Razzak, M.I. Handwritten Urdu character recognition using one-dimensional BLSTM
classifier. Neural Comput. Appl. 2019, 31, 1143–1151. [CrossRef]
\newline 
\newline
4. Husnain, M.; Saad Missen, M.; Mumtaz, S.; Jhanidr, M.Z.; Coustaty, M.; Luqman, M.M.; Ogier, J.-M.; Choi, G.S.
Recognition of urdu handwritten characters using convolutional neural network. Appl. Sci. 2019, 9, 2758. [CrossRef]
\newline 
\newline
5. Sarkhel, R.; Das, N.; Das, A.; Kundu, M.; Nasipuri, M. A multi-scale deep quad tree based feature extraction method
for the recognition of isolated handwritten characters of popular indic scripts. Pattern Recognit. 2017, 71, 78–93.
[CrossRef]\newline 
\newline
6. Xie, Z.; Sun, Z.; Jin, L.; Feng, Z.; Zhang, S. Fully convolutional recurrent network for handwritten chinese text
recognition. In Proceedings of the 23rd International Conference on Pattern Recognition (ICPR 2016), Cancun,
Mexico, 4–8 December 2016.
\newline 
\newline
7. Liu, C.; Yin, F.; Wang, D.; Wang, Q.-F. Online and offline handwritten Chinese character recognition: Benchmarking
on new databases. Pattern Recognit. 2013, 46, 155–162. [CrossRef]
\newline 
\newline
8. Wu, Y.-C.; Yin, F.; Liu, C.-L. Improving handwritten chinese text recognition using neural network language models
and convolutional neural network shape models. Pattern Recognit. 2017, 65, 251–264. [CrossRef]
\newline 
\newline
9. Gupta, A.; Sarkhel, R.; Das, N.; Kundu, M. Multiobjective optimization for recognition of isolated handwritten Indic
scripts. Pattern Recognit. Lett. 2019, 128, 318–325. [CrossRef]
\newline 
\newline
10. Nguyen, C.T.; Khuong, V.T.M.; Nguyen, H.T.; Nakagawa, M. CNN based spatial classification features for clustering
offline handwritten mathematical expressions. Pattern Recognit. Lett. 2019. [CrossRef]
\newline 
\newline
11. Ziran, Z.; Pic, X.; Innocenti, S.U.; Mugnai, D.; Marinai, S. Text alignment in early printed books combining deep
learning and dynamic programming. Pattern Recognit. Lett. 2020, 133, 109–115. [CrossRef]
\newline 
\newline
12. Ptucha, R.; Such, F.; Pillai, S.; Brokler, F.; Singh, V.; Hutkowski, P. Intelligent character recognition using fully
convolutional neural networks. Pattern Recognit. 2019, 88, 604–613. [CrossRef]
\newline 
\newline
13. Cui, H.; Bai, J. A new hyperparameters optimization method for convolutional neural networks. Pattern
Recognit. Lett. 2019, 125, 828–834. [CrossRef]
\newline 
\newline
14. Tso, W.W.; Burnak, B.; Pistikopoulos, E.N. HY-POP: Hyperparameter optimization of machine learning models
through parametric programming. Comput. Chem. Eng. 2020, 139, 106902. [CrossRef]
\newline 
\newline
15. He, K.; Zhang, X.; Ren, S.; Sun, J. Deep residual learning for image recognition. In Proceedings of the 2016 IEEE
Conference on Computer Vision and Pattern Recognition (CVPR), Las Vegas, NV, USA, 26 June–1 July 2016.
\newline 
\newline