\begin{center}
    \huge \textbf{Abstract}
\end{center}
\vspace{15mm}
    Identification of brain abnormalities is crucial for diagnosing neurological disorders and guiding individualized treatment strategies for patients. The goal of this project, "Unboxing an Industry-Level Deep Learning Model for Brain Anomaly Detection," is to thoroughly investigate the implementation and performance of a state-of-the-art deep learning model for brain anomaly detection in real-world applications. The study considers technical, ethical and other factors as it discusses the difficulties and obstacles to establishing an industry-level model. Research goals of the study include evaluating the performance of deep learning models on a large dataset of brain scans from several neuroimaging modalities. The aim of the study is to evaluate the accuracy, sensitivity and generalizability of the model through rigorous experiments and validation in different patient groups and clinical situations. The study aims to provide ethical principles to protect patient privacy and increase trust in healthcare solutions. This study is significant because it has the potential to completely change the way medical diagnostics and brain anomalies are detected. This work aims to revolutionize healthcare by unlocking the potential of an industrial-scale deep learning model to provide early detection and individualized care for patients with neurological diseases. The research aims to shed light on the technical complexity of implementing a deep learning model for brain abnormality detection at an industrial level in real healthcare settings. To successfully integrate with current healthcare systems, the research addresses challenges with data preprocessing, model training, and validation.