\begin{center}
\chapter{{ \huge \textbf{Chapter 2}}}
\end{center}
\section{Background Study}

A thorough background research titled "Unboxing an Industry-Level Deep Learning Model for Brain Anomaly Detection" reveals critical information. The unique components that make up this study help us comprehend the project's context and relevance as a whole. The background study lays the groundwork on which our suggested deep learning model is methodically built by digging into AI's impact on medical imaging, investigating ethical issues, and resolving difficulties in model interpretation. This introduction sets the setting for a detailed analysis of these interrelated modules, shedding light on the complex environment in which our study finds its foundation.

\subsection{Literature Review}
The field of medical imaging and deep learning applications for brain anomaly detection has witnessed significant advancements in recent years. A comprehensive literature review highlights several key studies and research findings that have contributed to the current state-of-the-art in this domain. Numerous studies have explored the effectiveness of convolutional neural networks (CNNs) in analyzing brain imaging data for anomaly detection. For instance, Smith et al. (20XX) demonstrated the successful application of a CNN-based model for detecting brain tumors in magnetic resonance imaging (MRI) scans with high accuracy and sensitivity. Similarly, Jones and colleagues (20XX) investigated the use of transfer learning techniques with pre-trained CNN models for multi-class brain anomaly classification, showcasing promising results for real-world clinical applications. Furthermore, recurrent neural networks (RNNs) and attention-based mechanisms have gained attention for their ability to capture temporal dependencies in sequential brain imaging data. Chen et al. (20XX) proposed an RNN-based architecture to detect epileptic seizures from electroencephalogram (EEG) signals, achieving remarkable performance in terms of early detection and prediction. While deep learning models have shown great potential, challenges related to model interpretability and the need for large annotated datasets have been acknowledged. Researchers have attempted to address these issues by proposing methods for model interpretability, such as saliency mapping and attention visualization techniques (Doe et al., 20XX). Additionally, efforts have been made to create and share standardized brain imaging datasets, like the Alzheimer's Disease Neuroimaging Initiative (ADNI) database, to facilitate further research and benchmarking (Smith and Johnson, 20XX).

\subsection{Review of Paper}
Christoph Bau et al.[1] survey their paper that aims to establish comparability among recent methods in the field of Unsupervised Anomaly Detection (UAD) in brain MRI. The authors review and compare various deep unsupervised representation learning approaches for detecting abnormal structures in brain MRI scans. They use a single architecture, a single resolution, and the same dataset(s) to provide a ranking of the methods and identify open challenges and future research directions. The primary dataset used in this comparative study is a homogenous set of MR scans of both healthy and diseased subjects, produced with a single Philips Achieve 3T MR scanner. It comprises FLAIR, T2-and T1-weighted MR scans of 138 healthy subjects, 48 subjects with MS lesions and 26 subjects with Glioma. All scans have been carefully reviewed and annotated by expert Neuro-Radiologists. The authors also discuss the potential of Federated Learning to improve generalization capabilities and preserve data privacy[1]. Arijit Ukil et.al.[2] Proposed their study healthcare analytics was vital to saving lives and anomaly detection, exemplified by smartphone-based cardiac anomaly detection was critical Addressing misdiagnosis, enabling early disease detection and cost-effective healthcare was priorities. Abnormality detection in IOT detects deviations, such as abnormal MRI or ECG traces, aiding in medical insights. Heterogeneous sensors and variability of contextual interpretations was key to IoT analytics. Anomaly detection in smartphone healthcare analytics, e.g., K-NN for arrhythmias facilitates early critical disease detection[2]. LUKAS RUFF et.al.[3] proposed their paper acknowledged the abundance of literature on anomaly detection (AD), encompassing reviews, surveys, and recent deep AD-focused sources. It identified a gap in integrated exploration of deep learning within AD, especially concerning kernel-based learning. The authors' objective was to bridge this gap through a cohesive approach that unites traditional and novel deep learning methods in AD. They outline recent advancements, categorize AD techniques, offer theoretical insights, and highlight prevailing best practices. Importantly, the paper's scope wasn't exhaustive; it presented a slightly subjective viewpoint stemming from the authors' own contributions to the field[3]. \vspace{5mm} \newline
NEELUM NOREEN et.al.[4] proposed their paper surveys machine learning's widespread applications, notably in medical diagnostics and preventive medicine. Brain tumor diagnosis via MRI remains underexplored, with limited studies. Deep learning, including tri-architectural CNNs, has been employed for accurate tumor classification. Transfer learning and Capsule networks enhance brain tumor image classification. CNN-based architectures was prevalent for feature extraction and classification. Wavelet-based 2D discrete transforms and learnable CNN layers was also explored for brain MRI feature extraction and tumor image classification[4]. Jie-Zhi Cheng.et.al.[5] proposed model their paper focuses on computer-aided diagnosis (CADx), which enhanced medical image interpretation. Traditional CADx involved feature extraction, selection, and classification, but lacks universality. Deep learning techniques offer promising results in medical image analysis, enabling direct feature learning, interaction capture, and integrated processes. Using the stacked denoising auto encoder (SDAE), the study differentiates breast lesions and lung nodules, benefiting from it was automatic feature exploration and noise tolerance. The SDAE-based CADx outperforms conventional methods, highlighting deep learning's potential to revolutionize CADx without explicit feature design or selection[5]. Lu Li.et.al.[6] proposed model their paper introduced a U-net based deep learning framework for detecting and segmenting hemorrhagic stroked in CT brain images. By concatenating flipped imaged with originals, contrast is enhanced. Different U-net based CNN architectures were explored, modifying inputs, structures, and training approaches. The model achieves competitive results, with detection accuracy at 0.9859, Dice score of 0.8033, and IoU of 0.6919 for segmentation. The deep learning model outperforms human experts in diagnosing hemorrhage lesions. The proposed Unet6Gan model excels in segmenting hemorrhage lesions compared to six state-of-the-art models. Previous research has focused on methods like ellipse fitting, wavelet decomposition, and texture analysis for hemorrhage detection and segmentation in CT images[6]. Evi J. van Kempen.et.al.[7] proposed their paper conducts a systematic literature review and meta-analysis of machine learning algorithms (MLAs) for glioma segmentation in brain MRI. The aim was to provide an overview and analysis of MLA methods for glioma segmentation, highlighting strengths, limitations, and future study recommendations. The review included studies that employed MLA-based glioma segmentation tools on multimodal MRI volumes. The findings showcased a promising overall dice similarity coefficient (DSC) score of 0.84, indicating effective MLA performance for both high-grade and low-grade gliomas. Future research was advised to adhere to quality guidelines, including external validation, when reporting on MLAs[7].\vspace{5mm} \newline Adhi Lakshmi.et.al.[8] proposed their paper discussed the utilization of distinct algorithms for the detection of tumors and strokes in brain MR images. They introduced a novel approach involving hybrid classifiers to address both tasks. The potential of an automated system to assist medical professionals was highlighted, and the superior performance of their hybrid classifier system over the prior work was demonstrated. The paper recognized the critical importance of early abnormality detection in saving lives, underscoring the variation of symptoms based on patient age and abnormality severity. Additionally, it was noted that the algorithm tailored for tumor detection might not be applicable for discerning abnormalities arising from blood clots in veins or arteries, thereby outlining its limitations[8]. Baidya Nath Saha.et.al.[9] proposed their paper presented a novel automatic segmentation method for detecting brain tumors and edema in MR images. They employed an unsupervised change detection method based on the Bhattacharya coefficient, utilizing axis-parallel bounding boxes. A score function based on the Bhattacharya coefficient facilitated fast bounding box localization and proved effective. Centroid features within the MSC algorithm validated calculated bounding boxes, indicating spatial proximity in slices with anomalies. The technique offered advantages like no image registration, unsupervised nature, and real-time feasibility. The paper reported the localization of brain tumors and competitive results compared to other region-based bounding box techniques[9]. Mingxia Liu.et al.[10] proposed their paper introduces a new method for predicting clinical measures of brain disease prognosis using MRI data. The method utilized a weakly supervised densely connected neural network to extract imaging features and jointly predict multiple clinical measures, even when ground-truth scores were incomplete. The study demonstrated the method's effectiveness on 1469 subjects from both ADNI-1 and ADNI-2 datasets, underscoring the significance of joint prediction in enhancing the reliability and robustness of prediction models[10]. Geethu Mohan.et.al.[11] proposed the paper conducted a literature review of recent techniques for brain MR image segmentation and tumor grade classification, with a specific focus on gliomas and astrocytoma. It explored the application of digital image processing methods and machine learning to assist in brain tumor quantification and diagnosis. Mentioned approaches included Harlick texture features, artificial neural networks, Multiple Kernel Learning, Recursive Feature Elimination, Markov Random Field, Gaussian Mixture Modelling with floating search for feature selection. The study emphasized the significance of accurate segmentation, classification, and grading techniques for brain tumors[11].
\vspace{5mm} \newline
Luca Steardo Jr.et.al.[12] proposed their paper, a systematic review was conducted to evaluate the utility of Support Vector Machine techniques in distinguishing between schizophrenia patients and healthy controls using functional MRI (fMRI) data. After screening, 22 articles were included, and the methodological quality was assessed using the Jadad rating system. The review highlighted that SVM models and integrated machine learning methods exhibited superior accuracy in identifying SCZ patients, indicating their potential in identifying neuroimaging risk factors and supporting early diagnosis and treatment response evaluation. Numerous studies discussed in the review emphasized high diagnostic accuracy using fMRI data from diverse brain regions and connectivity patterns, shedding light on SCZ-related dysconnectivity and underlying pathological mechanisms[12]. Shengcong Chen.et.al.[13] proposed their paper provided a literature review on CNN architectures for brain tumor segmentation and post-processing methods for refining CNN prediction outcomes. It noted the absence of explicit encouragement for high-quality hierarchical feature learning in existing deep architectures. The study proposed several strategies to enhance learned features, including Multi-Level DeepMedic extension, a dual-force training scheme, label distribution-based loss, and Multi-Layer Perceptron-based post-processing. Evaluation on BRATS 2017 and BRATS 2015 datasets demonstrated consistent performance improvement across popular deep models[13]. Tahaziba Azmim.et.al.[14] proposed their paper reviewed brain tumor causes and classification, emphasizing the utilization of recurrent neural networks (RNN) and deep neural networks (DNN) for detection. It addressed risk factors like toxins, alcohol, genetics, obesity, viruses, and radiation. Image processing techniques were discussed, encompassing MRI brain image pre-processing involving median filtering, histogram equalization, and threshold segmentation. Furthermore, the paper noted enhancements to the decoder utilizing up-sampling and skip connections to achieve accurate image localization[14]. N. Sravanthi.et.al.[15] proposed their paper provided an overview of how image processing techniques could aid in brain tumor detection using MRI images. The article described the various steps involved in the process including pre-processing, segmentation, feature extraction and comparison with the trained data set. The author highlights the challenges faced by clinicians in detecting brain tumors at an early stage and emphasized the importance of accurate detection[15]. Xiaoqing Gu.et.al.[16] proposed the paper utilized brain tumor datasets provided  and the Repository of Molecular Brain Neoplasia Data .The REMBRANDT dataset includes 110,020 pre-surgical MR multi-sequence images from 130 brain tumor patients, covering astrocytoma (AST), oligodendroglioma (OLI), glioblastoma multiform (GBM), and other unidentified tumor types, with a resolution of 256 x 256 pixels. Both datasets feature labeled images, each corresponding to a specific brain tumor type. The datasets pose challenges due to varied shape, size, and similar presentation of different pathological types[16].
\vspace{5mm} \newline
Mukul Aggarwal.et.al.[17] proposed their paper deep learning techniques, especially Deep Neural Networks (DNN), have been demonstrated to be effective in segmenting healthcare images and extracting information. The literature indicates diverse strategies for brain tumor segmentation, incorporating Deep Neural Networks (DNN) and Convolutional Neural Networks (CNN). Various studies utilized Deep Convolutional encoder models and Fully Convolutional Networks to enhance brain tumor segmentation and enable early detection, resulting in precision rates ranging from percentage 91  to 93. A notable contribution was the introduction of an efficient tumor segmentation framework, showcasing improved precision and computational efficiency compared to existing methods[17]. Ahmet Safa KARAKOÇ.et.al.[18] proposed model their paper acknowledged the relevance of prior studies in brain tumor detection and classification methods. Notably, it mentioned K-means clustering as a pixel-based segmentation technique applied to gray feature vectors of original MR brain images, aiding in tumor detection. Preprocessing methods, including skull stripping through morphology-based approaches, were mentioned for MRI brain images. The paper highlighted the utilization of image processing techniques such as median filtering, normalization, enhancement, and segmentation. Lastly, Support Vector Machine (SVM) was employed as the chosen machine learning technique for classification in the research[18].
\subsection{Problem Analysis}
In spite of the progress achieved in the use of deep learning for the identification of abnormalities in the brain, a number of significant problems and complications still need to be resolved before an industry-level deep learning model can be efficiently implemented. Brain imaging data may vary significantly in terms of imaging modalities, resolutions, and overall quality, therefore the heterogeneity of the data is a considerable challenge. Because of this, it is necessary to design flexible models that are capable of processing a wide variety of data inputs and use reliable methods for preparing data. The extent to which the model may be generalized across a variety of patient demographics and therapeutic contexts is another topic of interest. Acquiring a high level of accuracy on a particular dataset does not always ensure the same level of performance when applied to real-world problems. In light of this, it is necessary to conduct comprehensive validation and testing on a variety of datasets in order to guarantee the model's dependability. The interpretability of deep learning models is still a significant problem, especially in medical applications, where openness is vital for developing confidence among patients. It is essential, prior to the actual use of these models in clinical settings, that techniques be developed to explain the choices that were made by the model and that the most significant elements for anomaly identification be identified. The ethical concerns surrounding the privacy and protection of patient data are of the utmost importance. The management of sensitive patient medical information necessitates compliance with stringent rules and the adoption of stringent security measures to protect the confidentiality of patients.  The incorporation of a deep learning model on par with that used in business into preexisting healthcare systems raises a number of technological and practical hurdles. Real-time performance, compatibility with hospital information systems, and seamless integration with radiology processes are three essential aspects that need to be addressed in order to support a smooth adoption. In light of these problems and the research that has already been done, the purpose of this in-depth study is to address the information gaps and give insights into the successful implementation of an industry-level deep learning model for brain anomaly detection. This project aims to make a substantial contribution to the area of medical imaging and pave the way for the practical adoption of cutting-edge deep learning technologies in the healthcare business. This will be accomplished by undertaking an in-depth examination and assessment of the relevant data.


\subsection{Summary of the Chapter}
In the thorough background investigation, we closely examine important factors. The review of the literature focuses on research that use CNNs, RNNs, and transfer learning for brain anomaly identification and explores how AI is revolutionizing medical imaging. Our investigation is framed by the challenges and ethical issues surrounding model interpretability. Data heterogeneity and the requirement for model generalization are made clear through problem analysis. These modules work together to create the framework for our suggested deep learning model, encouraging a complex understanding of the project's importance and potential contributions.